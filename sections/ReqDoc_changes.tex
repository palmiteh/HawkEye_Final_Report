Before real development began, we had reread our requirements document and noticed that some changes and additions needed to be made. The revision was needed in order to refine our project with the discovery of problems/improvements we had found. It was a needed process as we wanted to have full clarity on what needed to be accomplished with this project, and be able to display those requirements with our sponsor. The changes were reviewed with our sponsor and agreed upon before any final changes were made.
\par
Only a few minor things were changed, but they had a big impact on how we define our deliverable according to their standard. Our client was very happy to see the revisions we had made and all final revisions were made in conjunction with our document. The table below shows the minor changes we had made.\\

\begin{center}
	\begin{tabular} { | c | c | c | p{5cm} | }
	\hline
	\multicolumn{ 4 } { | c | } {Requirements Document Changes} \\
	\hline
	1 & Two Working Cameras &  Added Requirement &  In our original document we had not specified more than one camera, but our sponsor specifically wanted to make that a requirement as the system must be able to handle more than one camera. Therefore it was added into our requirements document.\\ \hline
	2 & Use of Either Jetson TK1 or TX1  & Redefined Requirement  &  Before the revision, the requirements document was specifying the use of only the Jetson TK1. After much research on our end we had decided that either Jetson's TK1 or TX1 would be adequate for development, and actually plan to rely more heavily on the Jetson TX1. This change allows us to have more flexibility between the two Jetson SBC's, which provides protection if we cannot get one to work properly or to the meet the requirements needed.\\ 
	\hline
	\end{tabular}
\end{center}