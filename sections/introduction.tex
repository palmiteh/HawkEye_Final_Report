Rockwell Collins, our project sponsor, designs video vision systems for pilots to use during flight. Pilots often use these enhanced image systems to help them see better in rough weather conditions, and to generally assist during different flight operations such as landing. An example of this functionality could be overlaying a pilot's view with graphics to help the pilot locate a landing strip in a storm or automatically turning a night-vision camera off when it is not needed in order to save power. The hardware used for this system must limit the power consumption it pulls from the aircraft, and also must be lightweight as weight is a very expensive resource on aircraft. Even an extra five pounds added to the plane during a year can add up to thousands of dollars in fuel costs. In order to provide pilots with enhanced image processing while maintaining low power consumption and low weight, Rockwell Collins develops software on Field Programmable Gate Arrays (FPGAs). While it meets the power and weight requirements, this piece of hardware makes the code very complex and very costly to develop. New vision enhancements can take weeks or months to develop on the FPGAs. The FPGAs are currently Rockwell Collins only option that meets the requirements needed to create practical systems that pilots can use effectively.\\
\par
Our goal is to provide a proof of concept for an alternative to Rockwell Collins' FPGAs. It must meet the performance metrics, provide faster implementation time, and reduce the cost of production. Specifically, we are designing a proof of concept using single board computers (SBCs). Single board computers differ from FPGAs because they have a standardized execution environment, which allows simpler code to be executed on it, reducing the development time. Single board computers also use low cost hardware and don't consume much power. If we can prove that the video quality produced by the SBCs is adequate for pilots to use, they will meet all the requirements needed to develop a practical vision system. In the air, FPGAs often take feeds from multiple cameras and run a lot of processing algorithms on those images. In order to be effective, the SBCs should also be able to handle this operation. Our project aims to test and measure the capabilities of a single board computer by delivering a multiple-stream video display that has been processed to a high degree. We believe the best candidate for this is to use NVIDIA's single board computer: the Jetson TX1. Our goal is to fully test the Jetson TX1's ability to provide enhanced imaging. This proof of concept will result in measurements that will help Rockwell Collins determine the practicality of using single board computers for their vision systems.\\
\par
   %About clients roles and who they were, team members and our roles%
Our clients are Carlo Tiana and Weston Lahr through Rockwell Collins.
The members of the HawkEyed Crew are: Scott Griffy, Hailey Palmiter, and Ryan Kitchen.
Ryan Kitchen worked on the technical side of the project. Scott helped with some of the technical side and wrote documentation. Hailey edited and wrote documentation and managed the project.
We communicated with the clients to define requirements and secure hardware necessary for the project. We also got a lot of the hardware from our instructor, Kevin McGrath.
