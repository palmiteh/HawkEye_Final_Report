\subsubsection{What technical information did you learn?}
I learned a lot about the hardware of single board computers. I often work with graphics on desktops, but you don't have to dive into the hardware when you're working on a desktop, all the functionality is given to you through library calls and APIs. On the Jetson we literally had to recompile the kernel at times to get more functionality out of the board. I've never had to recompile a kernel to complete a project and knowing how to do so improved the technical tools I have at my disposal for computer projects.
\par
I also learned a lot on the software side of graphics processing. I learned about CUDA and the cuda libraries, especially how to maximize the performance when I'm working with graphics libraries. When I'm working on higher-level projects, most of the performance enhancements I make are in the algorithm, but when you're working with low level hardware, the difference is in the implementation. Memory copies are the enemy in latency for graphics systems and after we minimized those with custom buffers for the camera capture and GPU textures, we got the best latency. I also learned more about how camera capture APIs work and how to access the physical registers on a camera.
\par
I feel that I learned more technical stuff that will help me in future projects if needed, but I can't specifically recall it here.
\subsubsection{What non-technical information did you learn?}
\par
I think the biggest takeaway that I got out of this project was working with a team in a long term project like this. I've worked in groups on smaller projects before but never for a whole year. In some ways working with a group for a longer amount of time is easier and in some ways it's harder. In a short project, the difficulty is learning how to work with the group in the short amount of time. When you open it up to a larger project, you realize that group projects 
\par
I also learned about how to explain technical things to a non-techincal audience. This was important for some of the documents we wrote as well as at expo. Often, when I'm working on a project, I get caught up in the code and never boil it down so that I can explain it to people. This might be the most important part of any technical work, explaining it to other people so it can be used and improved upon.
\subsubsection{What have you learned about project work?}
\par
When you're working on a project, you need to make sure that you're on task and that the task you're doing is important. This is the most important part of project work and a lot of the effort in a project goes towards defining these tasks. If your project is well thought-out and you're on task, the project will get completed and be successful and if you've met and defined good requirements, then all you need to do when you're working on your own is stay on task. This might seem simple, but it's really easy to get off task and try to feature creep on your own, or end a meeting early deciding that you'll figure out the rest of the tasks on your own. These are both examples of bad project work and should be avoided.
\subsubsection{What have you learned about project management?}
\par
Setting up meetings and assigning work isn't always easy. Sometimes the correct path isn't clear, but you still have to make a decision. Sometimes you don't have time to weigh all the options and simply need to choose a route that might not be optimal. Management is about a lot of small things and corner cases. There's never a correct answer and you never make the best decisions. The important part is simply making decisions and not being entirely wrong. Because making an okay decision is a lot better than making no decision at all.
\subsubsection{What have you learned about working in teams?}
\par
When you're working in teams with people you have to worry about workloads. You don't want one person to do all the work while the others have too much free time. That's why the work needs to be split up.
\subsubsection{If you could do it all over, what would you do differently?}
\par
If I could do it all over again, I wouldn't change much. We really did the best we could at each stage of the project. Maybe I would've been more involved with the image processing algorithms, because I fell behind a bit and was playing catch up learning the vision processing algorithms at the end of the term.
