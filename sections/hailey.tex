\subsubsection{What technical information did you learn?}
I began this project with no real background in hardware and no background in vision systems to include graphics libraries. Therefore the entire technical side of our project was a learning experience for me. It started with the use of the Jetson TK1 and TX1 development environment that used Ubuntu. I have interacted with the Linux operating system in the Operating Systems classes. The PointGray camera had their own drivers we had to install and work with. PointGrey also included its own graphical user interface called FlyCapture that we had to manipulate. Thus we needed to understand what was happening within those packages and interface in order to have it do what we needed it to. 
\par
The language I got the most exposure with was OpenGL we had to interact with several of its libraries and its toolkit called GLUT. I also learned about CUDA and using CUDA kernels to handle filter operations on our captured image buffers. I learned how an image is captured from the camera and stored into a buffer, which we created into a shared buffer between the GPU and CPU. I also learned how to use parallel processing with image buffers to increase latency and frame rate speeds. 
\par
During the creating of the runway I was also able to work with a micro controller in order to program our LED strip lights to produce a runway like approach lighting system.\\

\subsubsection{What non-technical information did you learn?}
A few non-technical things that I learned involved creating latex documents, FAA runway standards, and creating formal documentation on our project. Using latex and the IEEEtran formatting style we were able to create very professional looking documents. There was a steep learning curve for latex, but once I had figured it out it made all of our documentation very simple to create. Even though I had learned about requirements documents, design documents, and technical reviews in the Software Engineering classes, it was a whole new learning experience creating them for a real industry like project. 
\par
Another non-technical experience was learning how interact and deal with a client. During the planning process of our project we had to have a lot more involvement with our client to make sure we were establishing all of the requirements they had for us and were also creating the project they had in mind. Later in development I had to keep them updated with our progress, set up meeting, create an outline for the meetings along with producing meeting minutes after it had concluded. Throughout the entire project we needed to assure that they were informed and happy with what we were doing for them, while also follow the requirements document we had established.\\

\subsubsection{What have you learned about project work?}%Finish
I have learned that there are a lot of moving parts to a project and that it is much

\subsubsection{What have you learned about project management?}
I have learned that project management runs much smoother once you are able to find out how other people work. Knowing your team, their strengths and weaknesses, availability, and how they work best is a piece of how to lead people well. I learned how to set each of my team members up for success by assigning tasks appropriately based on what they enjoyed, what they were good at, and the time they had to work on a specific task. 
\par
I also learned that staying organized and setting deadlines/goals for things help us all to stay focused and to complete the things we needed to on time.\\

\subsubsection{What have you learned about working in teams?}
I have learned that working in a team doesn't matter if you are all the best of friends, but what matters is being able to leave it all at the door to create a professional work environment where all entities are comfortable and feel as being an important piece of the project. I was lucky that our team all got a long well and worked really well together, but I believe it also taught me what a good team dynamic should look and feel like. Each of us had a strength to bring to the project, which created a very strong team dynamic. I also learned about compromising with others on ideas as well as being able to help support team members with tasks when needed.\\ 

\subsubsection{If you could do it all over, what would you do differently?}
