Unless otherwise specified, all code is in /HawkEyed\_Video/HawkEye\_2.0.\\
\par
The following code is the buffered camera system that is used in our final product.
The "Frame" type is a class we designed that maintains the data until it is done being used
and then releases it to be captured into again.\\
\lstinputlisting{code/camera_2.0.cpp}
\par
This code is the edge detection algorithm used in the demo. It uses a pixel and 3 of its neighbors
to determine the direction and magnitude of a change in brightness. This example shows how to develop a
CUDA kernel and some of the optimizations we used in order to reduce latency.\\
\lstinputlisting{code/edge_overlay.cu}
\par
This code is the main program for our demonstration application, which was used for the benchmarking
and at expo. It shows how different kernels can be strung together via streams, and the operations changed during
program execution.\\
\lstinputlisting{code/glut_output.cpp}
\par
This code is our pipeline system, which was developed and used to achieve even lower latency  for our software, but 
was not included in the final version because of stability issues, but would be the preferred method for future development.
It demonstrates a more effective way to process multiple frames at the same time, and optimizes to only grab the most
recent frames instead of processing all frames in order. It allows the programmer to avoid directly having to implement 
multithreaded, multi-stream processing, and instead focus on just the algorithm development.\\
\lstinputlisting{code/pipeline.h}