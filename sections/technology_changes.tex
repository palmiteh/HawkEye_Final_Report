\subsection{Technology Changes}
\subsubsection{Camera Interface}
For our camera interface we ended up using the USB 3.0 camera setup. This made our code simple, but did require us to work with a USB 3.0 expansion card which had trouble supplying the necessary power to the cameras, requiring the camera's power supplies to be connected while in use.
\subsubsection{Video Subsystem}
For our video software subsystem, we ended up using a different system entirely: CUDA. We talked about OpenCV's performance issues in the Technology Review and they ended up constraining us too much. OpenCV was great for getting a simple program working, but it only ran at around 14 frames per second which was not viable. Writing lower level code in CUDA kernels and using the CUDA API allowed us to take more control over the memory in the TX1 resulting in higher frame rate.
\subsubsection{Image Processing}
We ended up using the on-board GPU for our image processing which is what we thought we were going to use. It was the best option because it was the simplest to code and investigate and it supposedly would meet the requirements, so we had no reason to believe it wouldn't be capable for the project and if it failed we wouldn't have wasted much time.
\subsubsection{Demo Interface}
In our technology review, we discused a lot of software that we thought we might use to display the output of our camera system and decided to use OpenCV to create the project. We ended up using OpenGL instead. Our discussion of the benefits and drawbacks of OpenCV vs OpenGL in the technology review show why we eventually made the switch. The OpenCV GUI ended up being too slow compared to OpenGL. This was because of the benefit we list in OpenGL, saying that it was powerful. We had full control over the buffers that were send to the window for rendering, which allowed us to employ memory mapping magic to reduce the number of memory copies and thus achieve the latency and framerate required for the project. While this was more expensive in terms of coding hours, it ended up being necessary for the project. Though attempting to use OpenCV was a worthwhile effort to learn more about our development environment quickly.
