\subsubsection{October 16, 2105}
\paragraph{First Meeting and Problem Statement Developed}
This week consisted of our initial meeting with our sponsors (we found out that we more or less have two of them), and we were also able to construct a complete problem statement of the project. The meeting occurred Monday in the the Valley Library. Our sponsors were able to drive down from Rockwell Collins in Wilsonville to have a face-to-face meeting. It was very nice to be able to have this meeting and to clarify many of our questions/concerns about the project. After the meeting was completed, the team split up different bullet points of the problem statement to work on. A Google document was created, where we were all able to collaborate and work on the document. Once the content was created we went through and edited it all to have the same voice and proper language. The final document was then created as a PDF and sent our sponsor to be signed. They were very happy with the outcome and feel we were able to capture an accurate and complete vision of the project. During this week we also created the group project page and blog site, and were able to do a little research involving the capabilities of the NVIDIA Jetson board. This concluded the week for our project.?\\

\subsubsection{October 23, 2105}
\paragraph{Requirements Document}
This week we drafted a requirements document and sent it off to out client for review. We are planning to meet with him soon over a WebEx VOIP, which I've never used. Hopefully that works out well. We're going to get a signature for our requirements document for next week. We also fixed some issues with SharePoint this week. We accidentally created two sites and in a very tense moment, deleted one. Also getting everyone the correct permissions on SharePoint has been difficult.\\

\subsubsection{October 30, 2105}
\paragraph{Requirements Document Continued...}
This week overall has been a slow one, waiting for review and input. It took until later this week to get some feedback from our client about the rough draft of the requirements document, which was needed for further advancement on the document. The feedback was very helpful in letting us know that we need to narrow our requirements down, and focus on additives once we can finish the main requirements. We have begun to make a more defined document with added descriptions and better defined requirements. Our document will be completed and signed by Wednesday, November 4th.\\

\subsubsection{November 6, 2105}
\paragraph{Final Revision of Requirements Document}
This week, due to an extension, we were supposed to turn in our final draft of the requirements document signed by our sponsor. On Tuesday, the class was informed that many of our requirements documents were not up to standard. We now have until Tuesday of this coming week to rewrite our documents and submit them again. We were given an IEEE standards document of how to correctly structure a requirements document. We took this document as a group and restructured our requirements. We added a cover page, a table of contents, and much more description of the project as a whole and each individual task as well. On Wednesday, we had our first meeting with our TA, Xinze. He was very helpful in giving us feedback for our original document. Once we had updated the old document into the newer format, we sent him a copy of it and got even more feedback. Once the document was completed, and the Gantt chart created according to our requirements was added, we sent it to our customer for his input. We are still currently waiting on a response. I believe that we have a pretty solid requirements document now compared to our original attempt. We also have several other assignments coming up in the few weeks, like a technology review, an elevator pitch, and a rough draft of our poster.\\

\subsubsection{November 13, 2105}
\paragraph{Final Final Revision of the Requirements Document}
We revised the requirements document yet again and submitted an almost finished version to our class. We then were able to contact our client for edits and are working on finally finalizing it. We also wrote up the tech document. It looks like we're going to start out using OpenCV and see if we can setup up the system using that and then optimize it from there. Also we might switch to using the TX1, a newer version of the TK1.
\par
We're having a meeting with our client on Monday and should prepare for that.\\

\subsubsection{November 20, 2105}
\paragraph{Rockwell Collins Meeting and Poster Development}
On Monday this week, we met with Carlo and Weston from Rockwell Collins for our monthly check in. There were several items on the agenda:

\begin{itemize}[leftmargin=2cm,labelindent=2cm]
\item We went over the revisions we made to the Requirements document after receiving comments from Carlo via email last week
\item We discussed alternative single board computers, including the Sapphire Tech Step Eagle board currently being used by Rockwell Collins for a similar project as well as the new Jetson TX1
\item We formed a plan for evaluating camera options. Rockwell Collins is providing us with one USB 3.0 camera to test and see if we can get it to work with our Jetson. If it is not supported by the Jetson, we will begin looking at other options
\item We discussed our Gantt Chart and received recommendations from Carlo about additional milestones we could put in to add more specificity to our project plan. 
\item Carlo and Weston are still unable to connect to our sharepoint site, so Hailey is working with them to see if she can get them set up.
\item We discussed potential dates for our next meeting, which will be somewhere between December 14-16.  Depending on if we get the hardware to work together, we may meet in person either in Portland or Corvallis.
\end{itemize}
After our meeting, we also created a rough draft of our project poster for the Engineering Expo.\\

\subsubsection{November 27, 2105}
\paragraph{Thanksgiving}
We had the second half of thanksgiving off and so we weren't able to accomplish much more than planning out a few meetings for the next week.\\

\subsubsection{December 4, 2105}
\paragraph{Design Document and Progress Report}
This week we created a design document and progress report. The design document followed the IEEE Std 1016-2009 format, which we went through and met on Monday to start brain storming. Our main concern with this document was identifying our viewpoints and creating corresponding diagrams for each. Below gives an overview of our viewpoints and the corresponding design languages:

\begin{enumerate}[leftmargin=2cm,labelindent=2cm]
\item \textbf{Context viewpoint:}
This viewpoint shows the different potential users of the software and how they would
interact with our system.\\
Design Languages: Use Case Diagram
\item \textbf{Structure viewpoint:}
This viewpoint shows how the streaming video flows through the system, and identifies the
internal and external data connection points in the system. This viewpoint also shows the
components of the system which enable connection of cameras via USB as per the SRS.\\
Design Languages: Data Flow Diagram
\item \textbf{Interaction viewpoint:}
This viewpoint shows the order of operations on processing a video frame, as well as how
the timing is integrated to meet the performance benchmarking requirement.\\
Design Languages: UML Sequence Diagram
\item \textbf{Information viewpoint:}
This viewpoint details how the modular video processing system determines the order of
operations for the various elements and algorithms it can create. It shows how the system
meets the requirement for modularity and the capability to use multiple input and output
devices.\\
Design Languages: Entity Relationship Diagram
\item \textbf{State Dynamics Viewpoint:}
While the algorithm implementations in our design are stateless, the system itself is not.
This viewpoint shows the transitions between different states and provides a road map for
different states which will need to be individually tested.\\
Design Languages: UML State Transition Diagram
\end{enumerate}
The design document will be our roadmap to implementing our system to meet our requirements. We took special care in ensuring this document was done correctly as it is going to be a point of reference for the remainder of the year. We also created our progress report which covers a summary of the activities, problems, and solutions we came across during this Fall term. This winter break we will begin to familiarize ourselves with the Jetson development environment, and possibly connect our camera system to the board to find out how to connect the two environments.
